\section*{Abstract}
In today's world, technology is everywhere; we use it to communicate, to learn and to discover new things in order to improve our lives.
This paper presents a novel approach to the problem of augmenting the real world with digital content for entertaining or didactic purposes. We propose a new method to access digital content using \ac{QR} codes, present a proof of concept of our approach, and discuss the potential of \ac{AR} as an interactive tool for learning and discovering new things.
This thesis theme is the development of an Android mobile application called RealityEnhance. The application aims to augment the real world with digital content through the user's smartphone. RealityEnhance gives its users the power of an enhanced perception of their surroundings to discover and interact with digital content.
\newline
\newline
În lumea de astăzi, tehnologia este peste tot; o folosim pentru a comunica, pentru a învăța și pentru a descoperi lucruri noi pentru a ne îmbunătăți viața.
Această lucrare prezintă o abordare nouă a problemei îmbunătăţirii lumii reale cu conținut digital în scopuri de divertisment sau didactice. Propunem o nouă metodă de accesare a conținutului digital folosind coduri \ac{QR}, prezentăm o dovadă de concept a abordării noastre și discutăm potențialul realităţii augumentate ca instrument interactiv de învățare și descoperire de lucruri noi.
Urmărim dezvoltarea unei aplicații mobile pentru Android numită RealityEnhance. Aplicația își propune să îmbunatăţească lumea reală cu conținut digital prin intermediul smartphone-ului utilizatorului. RealityEnhance oferă utilizatorilor săi puterea unei percepții îmbunătățite asupra mediului înconjurător pentru a descoperi și a interacționa cu conținut digital.