\chapter{Introduction}\label{cap:introduction}

\section{Project Theme}
Augmented Reality is a technology combining the virtual and physical worlds, allowing digital content to be added on top of the physical world.
\ac{AR} relies on device sensors to detect the users' surroundings, position and orientation. \ac{QR} codes are a type of barcode; they are predominantly used to store information such as text, links, and images and are easy to share.

By combining these two technologies, we can create a new way of accessing digital content and interact with newly accessed information in the real world. This technology gives users the power of enhanced perception while accessing digital content that can be used for entertainment or didactic purposes.


\section{Theme Relevance}
The topic of \ac{AR} is relevant because it can be implemented and used in a multitude of domains. This technology is already used in the following domains:
\begin{itemize}
    \item \ac{3D} printing - \ac{AR} is used to visualise the \ac{3D} model of the object that will be printed.
    \item Architecture - \ac{AR}is used to visualise the \ac{3D} model of a building or a house.
    \item Automotive - \ac{AR} is used for navigation and to display map information.
    \item Publicity - \ac{AR} is used to lunch novel marketing campaigns and interact with digital content.
\end{itemize}


Augmented reality is an interactive experience that recently became more accessible to the world with the help of the smartphone. Most smartphones have the necessary hardware to run \ac{AR} applications. This \ac{AR} applications give their users a new perspective on the digital content that they are accessing.
\pagebreak

\ac{QR} codes are two-dimensional matrix barcodes that are easily scanned by a smartphone camera. They were first used for tracking parts in vehicle manufacturing. Nowadays, they are used in a wide variety of domains, such as:
\begin{itemize}
    \item Virtual stores - \ac{QR} codes are used to display information about the products.
    \item Online payments - \ac{QR} codes are used to store payment information.
    \item WI-FI access - \ac{QR} codes are used to store WI-FI access information.
    \item Augmented reality - \ac{QR} codes are used in some \ac{AR} applications to determine the positions of objects in \ac{3D} space.
\end{itemize}


\section{Project Aim}
The \ac{AR} topic was chosen because this technology has many applications in different fields. The immersive experiences of \ac{AR} allow people to do interactive activities.

The primary objective is to develop a highly functional mobile application that utilises \ac{AR} and allows users to scan \ac{QR} codes. Then using the available models in the \acf{AR} mode to move, resize, rotate and interact with them.

The application is developed for the Android platform. To accomplish this, Android Studio \ac{IDE} was utilized alongside Google's ARCore \ac{SDK}, which provides powerful augmented reality capabilities. In order to store the \ac{3D} models, we make use of the Google Buckets service, which offers a reliable and easy-to-use storage solution. This application has a user-friendly interface that is both simple and intuitive, ensuring that users can easily navigate and interact with all the features.

One of the key highlights of this application is its interactive functionality for some \ac{3D} models. Users can trigger engaging animations or explore different states of the model by simply touching a button. This innovative approach offers users a fresh way to access digital content and actively participate in the learning and discovery process. It opens up exciting new possibilities for users to create, learn and explore.

The application will undergo rigorous testing on multiple Android devices to ensure a seamless experience for all users. This will help identify and address any compatibility issues that may arise. Once the application has been thoroughly tested, it will be published on the Google Play Store, making it available to a broad audience.

\newpage
\section{Author's contribution}
I have utilized my knowledge and expertise to develop a mobile application that incorporates the Google ARCore \ac{SDK}, a \ac{QR} code scanner and the Google Buckets service. Through the integration of these technologies, I have created a novel approach to accessing and interacting with digital content. This application empowers users with an augmented perception of their surroundings, enabling them to engage with digital content using their smartphones.


The application is able to load the interaction module specific for each \ac{3D} model. The interaction module will be a simple animation or a different state of the \ac{3D} model. For example, if the \ac{3D} model is a LEGO car, the interaction module will be a simple guide for assembling the LEGO car. If the model is related to a geometrical shape, related to the school geometry curriculum, the interaction module will display a different perspective of the \ac{3D} model; for example, if the \ac{3D} model is a cube, the interaction module will display only the cube's edges.



