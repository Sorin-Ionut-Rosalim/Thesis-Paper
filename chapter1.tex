\chapter{Introduction}\label{cap:introduction}

\section{Project Theme}
Augmented Reality is a technology combining the virtual and physical worlds, allowing digital content to be added on top of the physical world.
\ac{AR} relies on device sensors to detect the users' surroundings, position and orientation. \ac{QR} codes are a type of barcode; they are predominantly used to store information such as text, links, and images and are easy to share.

By combining these two technologies, we can create a new way of accessing digital content and interact with newly accessed information in the real world. This technology gives users the power of enhanced perception while accessing digital content that can be used for entertainment or didactic purposes.


\section{Theme Relevance}
The topic of \ac{AR} is relevant because it can be implemented and used in a multitude of domains. This technology is already used in the following domains:
\begin{itemize}
    \item \ac{3D} printing - \ac{AR} is used to visualise the \ac{3D} model of the object that will be printed.
    \item Architecture - \ac{AR}is used to visualise the \ac{3D} model of a building or a house.
    \item Automotive - \ac{AR} is used for navigation and to display map information.
    \item Publicity - \ac{AR} is used to lunch novel marketing campaigns and interact with digital content.
\end{itemize}


Augmented reality is an interactive experience that recently became more accessible to the world with the help of the smartphone. Most smartphones have the necessary hardware to run \ac{AR} applications. This \ac{AR} applications give their users a new perspective on the digital content that they are accessing.
\pagebreak

\ac{QR} codes are two-dimensional matrix barcodes that are easily scanned by a smartphone camera. They were first used for tracking parts in vehicle manufacturing. Nowadays, they are used in a wide variety of domains, such as:
\begin{itemize}
    \item Virtual stores - \ac{QR} codes are used to display information about the products.
    \item Online payments - \ac{QR} codes are used to store payment information.
    \item WI-FI access - \ac{QR} codes are used to store WI-FI access information.
    \item Augmented reality - \ac{QR} codes are used in some \ac{AR} applications to determine the positions of objects in \ac{3D} space.
\end{itemize}


\section{Project Aim}
The AR topic was chosen because this technology has many applications in different fields. The immersive experiences of \ac{AR} allow people to do interactive activities.

The goal is to develop a highly functional mobile application that utilises \ac{AR} and allows users to scan \ac{QR} codes and interact with \ac{3D} models.

The application will be developed for the Android platform using the Android Studio IDE and Google's ARCore SDK. Also, the Google Buckets service will be used to store the \ac{3D} models and the \ac{QR} codes. It will have a simple and intuitive user interface and will provide a interactive function for some of the \ac{3D} models. The interactive function will be a simple animation or a different state of the \ac{3D} model that will be triggered by the user's touch input. It will provide users with a new way of accessing digital content and interacting with it, giving them the possibility to build, learn and discover the digital content that they are accessing. Also, the application will be tested on multiple Android devices to ensure compatibility and it will be published on the Google Play Store.


\section{Author's contribution}
My contribution is applying my knowledge and experience to develop a mobile application that uses the Google ARCore SDK module, a \ac{QR} code scanner and the Google Buckets service.


By integrating these technologies together, I have created a new way of accessing digital content and interacting with it. The application will give users the power of an enhanced perception of their surroundings to discover and interact with digital content through their smartphone.


The application is able to load the interaction module specific for each \ac{3D} model. The interaction module will be a simple animation or a different state of the \ac{3D} model. For example, if the \ac{3D} model is a LEGO car, the interaction module will be a simple guide for assembling the LEGO car. If the model is related to a geometrical shape, related to the school geometry curriculum, the interaction module will display a different perspective of the \ac{3D} model, for example, if the \ac{3D} model is a cube, the interaction module will display only the cube's edges.



