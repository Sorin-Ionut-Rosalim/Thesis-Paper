\chapter{Introduction}\label{cap:introduction}

\section{Project Theme}
Augmented Reality is a technology combining the virtual and physical worlds, allowing digital content to be added on top of the physical world.
\ac{AR} relies on device sensors to detect the users' surroundings, position and orientation. \ac{QR} codes are a type of barcode; they are predominantly used to store information such as text, links, and images and are easy to share.

By combining these two technologies, we can create a new way of accessing digital content and interact with newly accessed information in the real world. This technology gives users the power of enhanced perception while accessing digital content that can be used for entertainment or didactic purposes.


\section{Theme Relevance}
The topic of \ac{AR} is relevent because it can be implemented and used in a multitut of domains. This technology is already used in the following domains:
\begin{itemize}
    \item \ac{3D} printing - \ac{AR} is used to visualize the \ac{3D} model of the object that will be printed.
    \item Architecture - \ac{AR}is used to visualize the \ac{3D} model of a building or a house.
    \item Automotive - \ac{AR} is used for navigation and to display map information.
    \item Publicity - \ac{AR} is used to lunch novel marketing campaigns and interact with digital content.
\end{itemize}


Augmented reality is an interactive experience that recently became more accessible to the world by the help of the smartphone. Most of the smartphones have the necessary hardware to run \ac{AR} applications. This \ac{AR} applications give their users a new perspective on the digital content that they are accessing.
\pagebreak

\ac{QR} codes are a two-dimensinal matrix barcode that are easily scanned by a smartphone camera. They were first used for tracking parts in vehicle manufacturing. Nowadays, they are used in a wide variety of domains such as:
\begin{itemize}
    \item Virtual stores - \ac{QR} codes are used to display information about the products.
    \item Online payments - \ac{QR} codes are used to store payment information.
    \item WI-FI access - \ac{QR} codes are used to store WI-FI access information.
    \item Augmented reality - \ac{QR} codes are used in some \ac{AR} applications to determine the positions of objects in \ac{3D} space.
\end{itemize}


\section{Project Aim}
The \ac{AR} topic was choosen because this field is very immersing. This field of immersive experiences allows people to join enhancing activities.
The goal is to develop a working mobile application that allows users to scan a \ac{QR} code and use the smartphone's camera to visualize a \ac{3D} model of an object. They will be able to follow a guide and complete
%! TODO

\section{Author's contribution}
My contribution is applying my knowledge and experience to develop a mobile application that uses an \ac{AR} module for displaying and interacting with the digital content and a \ac{QR} code scanner module for scanning the \ac{QR} codes, loading the \ac{3D} models and displaying them with the \ac{AR} module.





