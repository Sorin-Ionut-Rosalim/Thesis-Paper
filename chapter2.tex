\chapter{Related Work}\label{cap:relatedwork}
\section{Related writings}
\subsection*{Augmented Reality in Education: Transforming Learning Experiences in the Classroom}

Introduction: In the realm of education, the integration of technology has revolutionized teaching and learning processes. This article explores the transformative potential of Augmented Reality (AR), and Quick Response (QR) codes in the context of education. Drawing inspiration from Monica Burns' book "Tasks Before Apps: Designing Rigorous Learning in a Tech-Rich Classroom," we delve into how AR and QR codes can enhance learning experiences and engage students in interactive educational content.

Augmented Reality: Immersive Learning at Your Fingertips Augmented Reality offers a unique opportunity to create immersive and interactive learning environments. By overlaying digital content onto real-world objects, AR provides students a dynamic and engaging platform for exploration. As Monica Burns highlights, AR can be likened to a "window to interactive learning" (Burns, 2017). Students can unlock a wealth of three-dimensional models, virtual objects, and multimedia resources through trigger images, such as posters or activity sheets. By scanning these triggers using smartphones or tablets, students are transported into a world where knowledge comes to life.

Integrating AR into the Classroom: Educators worldwide have embraced AR as a powerful tool for instructional design. Teachers can seamlessly integrate augmented reality into their lessons by leveraging AR apps and printing corresponding trigger images. For instance, using the Anatomy 4D app by DAQRI, students can explore the intricacies of the human body by scanning large posters of anatomical models. As Burns emphasizes, using AR captivates students' attention and provides a unique opportunity to interact with and deepen their understanding of content (Burns, 2017).

QR Codes: Unleashing Digital Resources Alongside AR, QR codes have become an efficient way to connect physical and digital resources. By scanning these codes using mobile devices, students gain instant access to a wealth of online content. With a simple scan, QR codes can link students to websites, videos, interactive quizzes, and other digital resources, expanding their learning beyond the confines of traditional materials.

Harnessing the Power of QR Codes: QR codes have become a versatile tool in the classroom, empowering teachers to deliver digital content seamlessly. Educators can create QR codes to supplement lessons, provide additional resources, or foster independent research. Students can embark on self-directed learning journeys by incorporating QR codes into lesson plans, exploring personalized content, and engaging in collaborative projects. The ease of use and immediate access to digital resources make QR codes invaluable in modern classrooms.
Synergistic Integration: AR and QR Codes Unite The combination of AR and QR codes unlock educational possibilities. Teachers can create synergistic learning experiences that blend the physical and digital realms by thoughtfully integrating both technologies. For example, students can scan a QR code to access background information or supplementary materials related to an AR experience, deepening their understanding and connecting with real-world applications.

Conclusion: Integrating Augmented Reality and QR codes in education holds tremendous potential for transforming learning experiences. Educators can foster engagement, deepen understanding, and inspire lifelong learners by immersing students in interactive virtual worlds and providing access to vast digital resources. As we embrace Monica Burns' notion of "tasks before apps," we must approach these technologies with clear pedagogical objectives (Burns, 2017). By strategically integrating AR and QR codes, we can create meaningful and impactful learning experiences that prepare students for success in a technology-driven world.
Reference: Burns, M. (2017). Tasks Before Apps: Designing Rigorous Learning in a Tech-Rich Classroom. ASCD.
\section{Related apps}

\subsection*{Snapchat AR}
In augmented reality (AR) applications, Snapchat AR stands out as a platform that revolutionises how we create, explore, and play. By harnessing the power of their AR Bar and a plethora of special Lenses, Snapchat AR enables users worldwide to scan their surroundings and discover valuable information. With just a tap on the Camera screen, the AR Bar unfurls, revealing myriad options, including the ability to create and edit unique Lenses.
What sets Snapchat AR apart is its commitment to accessibility. The introduction of the Web Lens Builder and Lens Studio tools has democratised the creation of AR Lenses, making it easier for aspiring creators to bring their visions to life. Moreover, Snapchat has embraced the incorporation of real-world physics and real-time data integrations, elevating the realism and immersion of their AR experiences to new heights.
Snapchat AR boasts a wide range of special Lenses, each offering a distinct and captivating visual experience. From playful animations to informative overlays, the creative possibilities are endless. Integrating real-world physics and real-time data ensures that the AR elements seamlessly interact with the user's environment, forging a more authentic and immersive encounter.
Despite its notable contributions, Snapchat AR does have its limitations. A consistent internet connection is imperative to fully enjoy the AR features, which may prove challenging in certain circumstances. Additionally, while Snapchat AR offers a compelling experience, it needs to improve in terms of functionality when compared to more specialised AR applications.

\subsection*{Pokemon Go}
In mobile gaming, Pokemon Go has become a global sensation by leveraging the power of augmented reality. The game's AR+ mode takes the concept of augmented reality to new heights by seamlessly merging the Pokemon universe with the player's real-world surroundings. Through this innovative approach, Pokemon come alive and appear anchored to the user's environment, providing exciting opportunities for capturing perfect photos and executing skilful throws.
In AR+ mode, Pokemon Go brings an added layer of realism to the gameplay experience. Players can approach, walk toward, or even move around these digital creatures by fixing Pokemon to specific points in the real world. The Pokemon in AR+ mode possess an uncanny awareness of the player's proximity and movement, requiring strategic and cautious approaches to maximise successful encounters.
In contrast, the AR mode in Pokemon Go detaches Pokemon from the real-world environment, lacking the anchor points and awareness of the player's location and movement found in AR+ mode.
The enhanced AR features in Pokemon offer players a captivating and dynamic experience. As they explore their surroundings, Pokemon seemingly inhabit their world, creating magical moments for excellent throws and captivating photos. The fusion of augmented reality with the beloved Pokemon franchise has captivated millions of players worldwide, fostering an engaging and interactive gameplay experience.
Like Snapchat AR, Pokemon Go relies on a consistent internet connection to enable the AR features. This requirement may pose challenges in situations where a stable connection is unavailable. Additionally, while Pokemon Go's AR+ mode excels in delivering a unique Pokemon experience, it may need to catch up when compared to dedicated AR applications that offer a broader range of functionalities.

\subsection*{ARFICIO}
While undertaking my bachelor's thesis, I discovered an intriguing app called ARFICIO. Although ARFICIO shares some similarities with the goals I aim to achieve, it operates on a partially free model and offers different features. It is worth noting that utilising ARFICIO requires a permanent internet connection.
ARFICIO provides users a partially free usage experience, making it accessible to a broader audience. However, in my research, I have observed that ARFICIO may have certain limitations regarding features compared to the specific objectives I seek to fulfil. Nevertheless, it serves as an essential point of reference and inspiration for my endeavours in augmented reality.
To obtain more comprehensive documentation on ARFICIO and other AR apps, I recommend conducting further research by referring to articles, reviews, and the official websites of these applications.
Incorporating these apps into your bachelor thesis showcases the impact and diversity of augmented reality applications. Snapchat AR revolutionises creative expression and immersion, Pokemon Go captivates gamers with its AR+ mode, and ARFICIO inspires your research journey. By examining these apps and their strengths and limitations, you gain valuable insights into the current landscape of augmented reality and pave the way for your innovative contributions in the field.